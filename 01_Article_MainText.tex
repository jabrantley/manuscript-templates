 \leadauthor{Scientist}

\title{Basic template for K-Lab journal submission}
\shorttitle{Running title here}

\author[1]{First Scientist \orcidlink{0000-0001-0000-0000}}
\author[2]{Second Doctor \orcidlink {000-0002-0000-0000}}
\author[1,\Letter]{Third Professor \orcidlink {000-0003-0000-0000}}
\affil[1]{Dept. of Bioengineering, University of Pennsylvania}
\affil[2]{Department of Neuroscience, University of Pennsylvania}
\date{}

\maketitle

\begin{abstract}
Abstract of the paper goes here.
\lipsum[1]
\end{abstract}

\begin{keywords}
keyword1 | keyword2 | keyword3
\end{keywords}

\begin{corrauthor}
third.professor\at upenn.edu
\end{corrauthor}

\section*{Introduction}\label{s:introduction}

The Kalman filter model assumes the true state at time $k$ is evolved from the state at $(k - 1)$ according to
\begin{equation}
    x _{k}=F_{k} x _{k-1}+ B_{k} u _{k}+ w,
\end{equation}
(or in bold font)
\begin{equation}
    \mathbf{x}_{k}=\mathbf{F}_{k} \mathbf{x}_{k-1} + \mathbf{B}_{k} \mathbf{u}_{k}+ \mathbf{w},
\end{equation}
where

\begin{itemize}
    \item $F_k$ is the state transition model which is applied to the previous state $x_{k-1}$;
    \item $B_k$ is the control-input model which is applied to the control vector $u_k$;
    \item $w_k$ is the process noise, which is assumed to be drawn from a zero mean multivariate normal distribution, $\mathbf{\mathcal{N}}$, with covariance, $\mathbf{Q_k}: {\displaystyle \mathbf {w} _{k}\sim {\mathcal {N}}\left(0,\mathbf {Q} _{k}\right)}$.
\end{itemize}

At time k an observation (or measurement) $z_k$ of the true state $x_k$ is made according to
\begin{equation}
    {\displaystyle \mathbf {z} _{k}=\mathbf {H} _{k}\mathbf {x} _{k}+\mathbf {v} _{k}},
\end{equation}

where
\begin{itemize}
    \item $H_k$ is the observation model, which maps the true state space into the observed space and
    \item $v_k$ is the observation noise, which is assumed to be zero mean Gaussian white noise with covariance ${\displaystyle R_k:  \mathbf {v} _{k}\sim {\mathcal {N}}\left(0,\mathbf {R} _{k}\right)}$.
\end{itemize}
% 
% 
And that is how rockets work. Boom! \lipsum[6-10]

\section*{Results}\label{s:results}

\subsection*{Citations and full size figures with legends underneath}

Text is added like this
This is a reference to a published paper \citep{watson_molecular_1953}.
We can cite other things too \citep{tipton_complexities_2019,zheng_genome_2011,alberts_molecular_2002}

This is a new paragraph.
New sentences on a new line.
New sentences on a new line.

% this is how to add a comment
This is a new result.
% this is how to add a figure with the name cells.
As you can see (Figure \ref{fig:cells}).

% full size figure is figure*
\begin{figure*}
\centering
\includegraphics[width=0.75\linewidth]{Figures/temp.png}
\caption{\textbf{These are cells.}\\
(\textbf{A}) This is a regular figure with a legend as a caption underneath. Inset: 3X zoom. Scale bar, \SI{10}{\micro\meter}.}
\label{fig:cells}
\end{figure*}

It is possible to add a one-column Figure like this (Figure \ref{fig:nucleus}).
To add Supplementary Figures you can do either of these things and have them at the end of the end of the paper (Supplementary Figure \ref{suppfig:endosome}).
Or like this (Supplementary Figure \ref{videosupp:lysosome}).

\lipsum[10]

\subsection*{Subsections are written like this}

\lipsum[11]

% one-column size figure is figure
\begin{figure}
\centering
\includegraphics[width=0.75\linewidth]{Figures/temp.png}
\caption{\textbf{This is a nucleus.}\\
(\textbf{A}) This is a one-column figure with a legend as a caption underneath.}
\label{fig:nucleus}
\end{figure}

\lipsum[12]

\subsection*{Another subsection}

\lipsum[13-14]

\subsection*{Another subsection}

\lipsum[13-14]

\subsection*{Another subsection}

\lipsum[13-14]

\section*{Discussion}\label{s:discussion}

This is the discussion section where you wax lyrical about your findings.
You can put your work in the context of other published work \citep{brenner_uga:_1967}.

\lipsum[100-104]

\section*{Methods}\label{s:methods}

\subsection*{Molecular biology}

Details of plasmids are usually first.
Followed by cell biology section.
We have special units defied for molar and for units, e.g. \SI{1}{\Molar} sucrose, \SI{10}{\Units\per\milli\litre}.
Otherwise use siunitx for everything else. \SI{37}{\degreeCelsius} and what-not.

\subsection*{Cell biology}

\lipsum[80]

\section*{Bibliography}
\bibliographystyle{bxv_abbrvnat}
\bibliography{refs.bib}